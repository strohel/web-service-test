\documentclass[12pt]{beamer}

\usetheme{Air}
\usepackage{thumbpdf}
\usepackage{wasysym}
\usepackage{ucs}
\usepackage[utf8]{inputenc}
\usepackage{pgfarrows,pgfnodes,pgfautomata,pgfheaps,pgfshade}
\usepackage{verbatim}

\pdfinfo
{
  /Title       (Web Services)
  /Creator     (TeX)
  /Author      (Matěj Laitl, Matěj Novotný)
}


\title{Web Services}
\subtitle{Aby byly strojově čitelné zprávy čitelné i pro vás}
\author{Matěj Laitl, Matěj Novotný}
\date{4. listopadu 2010}

\begin{document}

\frame{\titlepage}

\section*{}
\begin{frame}
  \frametitle{Co si dnes povíme}
  \tableofcontents[section=1,hidesubsections]
\end{frame}

\AtBeginSection[]
{
  \frame<handout:0>
  {
    \frametitle{Přehled}
    \tableofcontents[currentsection,hideallsubsections]
  }
}

\AtBeginSubsection[]
{
  \frame<handout:0>
  {
    \frametitle{Přehled}
    \tableofcontents[sectionstyle=show/hide,subsectionstyle=show/shaded/hide]
  }
}

\newcommand<>{\highlighton}[1]{%
  \alt#2{\structure{#1}}{{#1}}
}

\newcommand{\icon}[1]{\pgfimage[height=1em]{#1}}



%%%%%%%%%%%%%%%%%%%%%%%%%%%%%%%%%%%%%%%%%
%%%%%%%%%% Content starts here %%%%%%%%%%
%%%%%%%%%%%%%%%%%%%%%%%%%%%%%%%%%%%%%%%%%



\section{Web Services}

\begin{frame}
  \frametitle{Co jsou Web Services}
  \begin{alertblock}{Web Services nejsou:}
    \begin{itemize}
      \item Prohlížení webu skrz webový prohlížeč
      \item Na web zaměřené služby typu webhosting
    \end{itemize}
  \end{alertblock}
\end{frame}

\begin{frame}
  \frametitle{Co jsou Web Services}

  \begin{block}{Web Services jsou:}
    \begin{itemize}
      \item "Mechanismus pro výměnu strojově čitelných dat po síti mezi klientem a serverem"
    \end{itemize}
  \end{block}
\end{frame}

%%%%%%%%%%%%%%%%%% PUVODNI KOD NIZE %%%%%%%%%%%%%%%%%%%

\begin{frame}
  \frametitle{Prerequisites \& Goals}
  \framesubtitle{Knowledge is a brick wall that you raise line by line forever}
  \begin{block}{LaTeX}
  \begin{itemize}
    \item Obviously some basic LaTeX knowledge is necessary
    \item Some more features will be provided here
  \end{itemize}
  \end{block}

  \begin{block}{Beamer}
  \begin{itemize}
    \item You'll learn them by looking at this presentation source
  \end{itemize}
  \end{block}

  \begin{block}{Goal}
  \begin{itemize}
    \item Learn how to make well structured slides
    \item Using a beautiful theme (congrats to the Oxygen team!)
    \item Take over the world
    \item Relax...
  \end{itemize}
  \end{block}
\end{frame}

\section{Basic structuring}
\begin{frame}
  \frametitle{Sections, Frames and Blocks}
  \framesubtitle{Put everything into boxes}

  The current section is "Basic structuring". And the current frame
  is what you have on the screen right now.

  \begin{block}{A beautiful block}
  A block has a title, and some content. You can put in a block
  almost everything you want that is provided by LaTeX. For example
  math works as usual:
    \begin{equation}
    \sum_{i=1}^n i = \frac{n \times (n+1)}{2}
    \end{equation}
  \end{block}

  Also works outside a block:
  \begin{equation}
  \sum_{i=1}^n i^2 = \frac{n \times (n+1) \times (2n+1)}{6}
  \end{equation}
\end{frame}

\begin{frame}
  \frametitle{Different type of blocks}
  \framesubtitle{Weeeee! Colors!!}
  \begin{block}{Standard block}
  \begin{itemize}
    \item A standard block, used for grouping
    \item Obviously can contain itemizes too...
    \begin{itemize}
      \item And nested itemizes...
      \item of course!
    \end{itemize}
  \end{itemize}
  \end{block}
  \begin{alertblock}{Alert block}
  WARNING: Something very important inside this block!
  \end{alertblock}
  \begin{example}
  Note that examples are displayed as a special block...
  \end{example}
\end{frame}

\section{Fancy features}
\begin{frame}
  \frametitle{Highlighting}
  \framesubtitle{Hey! Look here! here!}

  \begin{block}{A regular block}
  \begin{itemize}
    \item Normal text
    \item \highlighton{Highlighted text} to draw attention
    \item \alert{"Alert'ed" text} to spot very important information
    \item Alternatively you can
    \begin{itemize}
      \alert{\item "Alert" the item itself}
      \highlighton{\item Or "Highlight" it}
    \end{itemize}
  \end{itemize}
  \end{block}
  \begin{alertblock}{If it's very very important...}
  \alert{... you can "alert" in an "alertblock"}\\
  Ewww, nasty, heh?
  \end{alertblock}
\end{frame}

\newcommand{\putlink}[1]{%
   \pgfsetlinewidth{1.4pt}%
   \pgfsetendarrow{\pgfarrowtriangle{4pt}}%
   \pgfline{\pgfxy(1,1)}{\pgfxy(#1,1)}
}

\begin{frame}
  \frametitle{Overlay effects}
  \framesubtitle{Keep the suspense!}
  \begin{block}{Time bomb}
  \begin{enumerate}
    \item<2-> Two more to go
    \item<3-> One more to go
    \item<4-> Last chance...
    \item<5-> BOOM!
  \end{enumerate}
  \end{block}
  \begin{block}{"Animation"}<6->
    \begin{pgfpicture}{0cm}{0cm}{7cm}{2cm}
    \only<1-6>{
      \putlink{2}
    }
    \only<7>{
      \putlink{4}
    }
    \only<8>{
      \putlink{6}
    }
    \only<9>{
      \putlink{8}
    }
    \only<10>{
      \putlink{10}
    }
    \end{pgfpicture}
  \end{block}
\end{frame}

\section*{}
\frame{
  \vfill
  \centering
  \highlighton{
  \usebeamerfont*{frametitle}And now?

  \usebeamerfont*{framesubtitle}Enter the secret section
  }
  \vfill
}
\begin{frame}
  \frametitle{Contributing to this beamer style}
  \framesubtitle{We want you !}

  \begin{block}{Why?}
  \begin{itemize}
    \item Beamer is hot!
    \item This style deserves to be improved
  \end{itemize}
  \end{block}

  \begin{block}{How?}
  \begin{itemize}
    \item Grab it
    \item Improve its LaTeX code
    \item Use you artistics skills
    \item Document it
    \item Help other people to use it
    \item Use it...
  \end{itemize}
  \end{block}
\end{frame}

\begin{frame}
  \frametitle{Resources}
  \framesubtitle{If you want to improve this style}
  \begin{thebibliography}{10}

  \beamertemplatearticlebibitems

  \bibitem{beamer-homepage}
    LaTeX Beamer
    \newblock {\tt http://latex-beamer.sourceforge.net/}

  \bibitem{kdeslides}
    KDE Presentations
    \newblock {\tt http://www.kde.org/kdeslides/}

  \end{thebibliography}
\end{frame}

%%%%%%%%%%%%%%%%%%%%%%%%%%% PUVODNI KOD VYSE %%%%%%%%%%%%%%%%%%%%

\frame{
  \vspace{2cm}
  {\huge Otázky?}

  \vspace{3cm}
  \begin{flushright}
    Matěj Laitl, Matěj Novotný

    \structure{\footnotesize{matej@laitl.cz, mates.novotny@gmail.com}}
  \end{flushright}
}

\end{document}
